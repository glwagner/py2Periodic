\documentclass[12pt, oneside]{article}

\usepackage{float}
\usepackage{lineno}
\usepackage{color, amssymb, amsmath, amsthm, verbatim, wasysym}
\usepackage{natbib}
\usepackage{epsfig}
\usepackage[mathscr]{eucal}
\usepackage{mathrsfs}
\usepackage{appendix}
\raggedbottom
\usepackage[left=0.8in,right=0.8in,top=0.8in,bottom=0.9in,centering]{geometry}      

% Uncomment to show references.
%\usepackage[notcite,notref]{showkeys}

% To make really wide hats that cover everything:
\usepackage{scalerel}
\usepackage{stackengine}
\setstackEOL{\#}
\stackMath
\def\hatgap{1pt}
\def\subdown{-0.2pt}
\newcommand\reallywidehat[2][]{%
\renewcommand\stackalignment{l}%
\stackon[\hatgap]{#2}{%
\stretchto{%
    \scalerel*[\widthof{$#2$}]{\kern-.6pt\bigwedge\kern-.6pt}%
    {\rule[-\textheight/2]{1ex}{\textheight}}%WIDTH-LIMITED BIG WEDGE
}{0.6ex}% THIS SQUEEZES THE WEDGE TO 0.5ex HEIGHT
_{\smash{\belowbaseline[\subdown]{\scriptstyle#1}}}%
}}

% For a 'strut' that creates space above underbraces
\newcommand*\mystrut[1]{\vrule width0pt height0pt depth#1\relax}

% Punctuation
\newcommand{\com}{\, ,}
\newcommand{\per}{\, .}

% A nice 'definition'
\newcommand{\defn}{\ensuremath{\stackrel{\mathrm{def}}{=}}}

% Use \bar to over line solo symbols
\newcommand{\av}[1]{\left \langle{#1} \right \rangle}
\newcommand{\avbg}[1]{\overline{#1}}
\newcommand{\avbgg}[1]{\overline{#1}}
\newcommand{\hav}[1]{\widehat{#1}}

% Begin and end equations
\newcommand{\beq}{\begin{equation}}
\newcommand{\eeq}{\end{equation}}

% Vector calculus operators
\newcommand{\p}{\partial}
\newcommand{\bnabla}{\boldsymbol \nabla}
\newcommand{\pnabla}{\boldsymbol \nabla_{\! \! \perp}}
\newcommand{\hnabla}{\bnabla_{\! \! h}}
\newcommand{\bnablad}{\bnabla_{\! \! \alpha}}
\newcommand{\bcdot}{\boldsymbol \cdot}
\newcommand{\hlap}{\triangle_h}
\newcommand{\lap}{\triangle}
\newcommand{\grad}{\bnabla}
\newcommand{\curl}{\bnabla \!\times\!}
\newcommand{\diver}{\bnabla \bcdot }
\newcommand{\cross}{\times}

% Bold symbolds
\newcommand{\bu}{\boldsymbol u}
\newcommand{\buh}{\boldsymbol u_h}
\newcommand{\bx}{\boldsymbol x}
\newcommand{\ba}{\boldsymbol{a}}
\newcommand{\bk}{\boldsymbol{k}}
\newcommand{\bh}{\boldsymbol{h}}
\newcommand{\bm}{\boldsymbol{m}}
\newcommand{\bn}{\boldsymbol{\hat n}}
\newcommand{\bxh}{\hspace{0.1em} \boldsymbol{\hat x}}
\newcommand{\byh}{\hspace{0.1em}\boldsymbol{\hat y}}
\newcommand{\bzh}{\hspace{0.1em}\boldsymbol{\hat z}}
\newcommand{\bnh}{\hspace{0.1em}\boldsymbol{\hat n}}
\newcommand{\bomega}{\boldsymbol \omega}
\newcommand{\bOmega}{\boldsymbol \Omega}
\newcommand{\bxi}{\ensuremath {\boldsymbol {\xi}}}
\newcommand{\bXi}{\ensuremath {\boldsymbol {\Xi}}}
\newcommand{\bU}{\boldsymbol{U}}
\newcommand{\bX}{\boldsymbol{X}}

% Greek abbrevs
\newcommand{\ep}{\epsilon}
\newcommand{\om}{\omega}
\newcommand{\kap}{\kappa}

% Roman characters
\newcommand{\ee}{\mathrm{e}}
\newcommand{\ii}{\mathrm{i}}
\newcommand{\cc}{\mathrm{cc}}
\newcommand{\dd}{{\rm d}}
\newcommand{\id}{{\, \rm d}}
\newcommand{\DD}{{\rm D}}
\newcommand{\J}{\mathrm{J}}
\renewcommand{\L}{\mathrm{L}}

% Non-dimensional numbers 
\newcommand{\Ri}{Ri}
\newcommand{\Ro}{Ro}
\newcommand{\Bu}{Bu}
\newcommand{\Pe}{Pe}

% Material derivative
\newcommand{\Dt}[1]{\mathrm{D}_t #1}

% Small in-line fractions
\newcommand{\half}{\tfrac{1}{2}}

% Bold 'F' for 'forcing'
\newcommand{\bff}{\boldsymbol{F}}
\newcommand{\fh}{\breve f}

% Dissipation operator
\newcommand{\friction}{\mathrm{F}}
\newcommand{\mixing}{\mathrm{M}}

\newcommand{\mode}{\phi}

\begin{document}

\title{\vspace{-4ex} Reduced equations for Boussinesq internal waves}
\author{Greg}
\date{} \maketitle \vspace{-4ex}

\section{The YBJ equation}

The two prognostic variables in the YBJ equation are wave field amplitude $A$ and quasi-geostrophic potential vorticity $q$. Quasi-geostrophic potential vorticity is related to the quasi-geostrophic streamfunction through the elliptic equation
\beq
q = \Big ( \underbrace{\mystrut{2ex} \p_x^2 + \p_y^2}_{\defn \hlap} + \underbrace{\mystrut{2ex} \p_z \frac{f_0^2}{N^2} \p_z}_{\defn \L} \Big ) \psi \com
\eeq
where the streamfunction is related to the velocity and buoyancy fields via $\bU = - \psi_y \bxh + \psi_x \byh$ and $B = f_0 \psi_z$.  The quasi-geostrophic potential vorticity solves the advection equation
\beq
q_t + \J \left ( \psi, q \right ) = \friction_u \hlap \psi + \mixing_b \L \psi \per
\eeq
The wave field amplitude represents the complexified horizontal velocity via
\beq
\tilde u + \ii \tilde v = \ee^{- \ii f_0 t} \L A \com
\eeq
and solves the YBJ equation
\beq
\L A_t + \frac{\ii f_0}{2} \hlap A +  \J \left ( \psi, \L A \right ) + \L A \left ( \tfrac{\ii}{2} \hlap \psi + \beta y \right ) = \friction_A \L A \per
\eeq

\subsection{In $x,y$ with barotropic $\psi(x, y, t)$}

\subsection{In $x, z$ with stationary $\psi(x, z)$}

\appendix

\section{The vertical mode decomposition}
\label{verticalModeProjection}

The hydrostatic vertical modes $\mode_n(z)$ solve the eigenproblem
\beq
\frac{f_0^2}{N^2} \mode_{nzz} + \lambda_n^{-2} \mode_n = 0 \com \qquad \text{with} \qquad \mode_n = 0 \quad \text{at} \quad z = -H, 0 \per
\label{modalEigenproblem}
\eeq
Note that the derivative $h_{nz}$ satisfies $h_{nz} = - \lambda_n^2 \L h_{nz}$.  The amplitudes of certain quantities are determined by their weighted projection onto $\mode_n$ or its derivative $\mode_{nz}$. For example, 
\beq
u_n \defn \int_{-H}^0 \Phi \, \mode_{nz} \id z \com \qquad b_n \defn \int_{-H}^0 b \, \mode_n \id z  \qquad \text{and} \qquad w_n \defn \int_{-H}^0 \frac{N^2}{\lambda_n^2 f_0^2} \, w \, \mode_n \id z  \per
\label{modezDef}
\eeq
The proper choice depends on the equation set and chosen boundary conditions.

Some trickiness is associated with the friction operator $\friction$. We want to assume, for example, that we can define an operator $\friction_n$ such that
\beq
\friction_{nu} (u_n) \approx \int_{-H}^0 \phi_{nz} \friction_u(u) \id z \com
\label{approximateFriction}
\eeq
for example. This is not always possible, however.  Notice that if $\friction = \nu \lap$, then
\beq
\int_{-H}^0 \phi_{nz} \nu \lap u \id z = \nu \hlap u_n + \frac{\kappa_n^2}{f_0^2} \int_{-H}^0 N^2 \phi_n u_{z} \id z \com
\eeq
and different modes are therefore coupled by the rightmost term. When $N$ is constant, however, this becomes 
\beq
\int_{-H}^0 \phi_{nz} \nu \lap u \id z = \nu \hlap u_n + \nu \left ( \tfrac{n \pi}{H} \right )^2 u_n \com
\eeq
and therefore $\friction_n = \nu \left ( \hlap + \left ( \tfrac{n \pi}{H} \right )^2 \right )$. A similar conundrum is associated with the diffusion operator $\mixing$.

\appendix

\section{The non-hydrostatic Boussinesq equations}

The rotating Boussinesq equations are
\begin{align}
\Dt{\bu} + 2 \bOmega \times \bu - b \bzh + \bnabla p &= \friction \bu \com \label{mom} \\
\Dt{b} + w N^2 &= \mixing b \com \label{buoy} \\
\bnabla \bcdot \bu &= 0 \com \label{cont}
\end{align}
where $\Dt \defn \p_t + \bu \bcdot \bnabla$ is the material derivative, 
\beq
2 \bOmega \defn \underbrace{2 \Omega \cos \phi}_{\defn \breve f} \byh + \underbrace{2 \Omega \sin \phi}_{\defn f} \bzh \com
\eeq
is the axis around which the Earth rotates, and the $\friction$ and $\mixing$ are operators that represent dissipative frictional processes and diffusive mixing processes, respectively. If dissipation and diffusion are due to isotropic molecular processes, then $\friction = \nu \lap$ and $\mixing = \kap \lap$, where $\lap = \p_x^2 + \p_y^2 + \p_z^2$ is the three-dimensional Laplacian. 

\section{`Wave operator form' of the Boussinesq equations}
\label{waveOperatorForm}

The component-wise rotating inviscid Boussinesq equations are
\begin{align}
u_t - f v + \fh w + p_x &= - \bu \bcdot \bnabla u \com \label{xmom} \\
v_t + f u + p_y &= - \bu \bcdot \bnabla v \com \label{ymom} \\
w_t - \fh u - b + p_z &= - \bu \bcdot \bnabla w \com \label{zmom} \\
b_t + w N^2 &= - \bu \bcdot \bnabla b \com \label{buoyComp} \\
\bnabla \bcdot \bu &= 0 \label{contComp} \per
\end{align}
We first form the `oscillation equation' with the combination $\p_t \eqref{zmom} + \eqref{buoyComp}$:
\beq
\left ( \p_t^2 + N^2 \right ) w - \fh u_t + p_{zt} = - \p_t \left ( \bu \bcdot \bnabla w \right ) - \bu \bcdot \bnabla b \per 
\label{oscillation}
\eeq
The 'divergence equation' follows from $- \p_x \eqref{xmom}- \p_y \eqref{ymom}$ and using $u_x + v_y = - w_z$, 
\beq
w_{zt} + f \omega - u f_y  -  \fh w_x - \hlap p =  \p_x \big ( \bu \bcdot \bnabla u \big ) + \p_y \left ( \bu \bcdot \bnabla v \right ) \per
\label{divergence}
\eeq
The vertical vorticity equation is obtained from $\p_x \eqref{ymom} - \p_y \eqref{xmom}$, 
\beq
\omega_t - f w_z + v f_y = - \pnabla \bcdot \left ( \bu \bcdot \bnabla \right ) \bu \per
\label{vorticity}
\eeq
Yes! In the penultimate step we calculate $\p_z \p_t \eqref{divergence} - f \p_z \eqref{vorticity}$, yielding
\beq
\Big [  \left ( \p_t^2 + f^2 \right ) \p_z^2 - \fh \p_x \p_z \p_t \Big ] w - \hlap p_{zt} = f_y \p_z \left ( u_t + f v \right ) + \p_z \left ( \p_t \bnabla + f \pnabla \right ) \bcdot \left ( \bu \bcdot \bnabla \right ) \buh \per
\label{divVort}
\eeq
Finally, the combination $\hlap \eqref{oscillation} + \eqref{divVort}$ yields the wave operator form,
\beq
\begin{split}
\Big [&  \lap \p_t^2 + f^2 \p_z + N^2 \hlap \Big ] w = \fh \p_t \left ( \hlap u + w_{xz} \right ) + u_t \fh_{yy} + 2 u_{yt} \fh_{y}  + f_y \p_z \left ( u_t + f v \right ) \\
& \qquad  + \p_z \left ( \p_t \bnabla + f \pnabla \right ) \bcdot \left ( \bu \bcdot \bnabla \right ) \bu - \hlap \left ( \bu \bcdot \bnabla b \right ) - \lap \p_t \left ( \bu \bcdot \bnabla w \right ) \per
\end{split}
\label{waveOperatorEqn}
\eeq

\bibliographystyle{jfm}
\bibliography{refs}

\end{document}